\PassOptionsToPackage{unicode=true}{hyperref} % options for packages loaded elsewhere
\PassOptionsToPackage{hyphens}{url}
%
\documentclass[
]{article}
\usepackage{lmodern}
\usepackage{amssymb,amsmath}
\usepackage{ifxetex,ifluatex}
\ifnum 0\ifxetex 1\fi\ifluatex 1\fi=0 % if pdftex
  \usepackage[T1]{fontenc}
  \usepackage[utf8]{inputenc}
  \usepackage{textcomp} % provides euro and other symbols
\else % if luatex or xelatex
  \usepackage{unicode-math}
  \defaultfontfeatures{Scale=MatchLowercase}
  \defaultfontfeatures[\rmfamily]{Ligatures=TeX,Scale=1}
\fi
% use upquote if available, for straight quotes in verbatim environments
\IfFileExists{upquote.sty}{\usepackage{upquote}}{}
\IfFileExists{microtype.sty}{% use microtype if available
  \usepackage[]{microtype}
  \UseMicrotypeSet[protrusion]{basicmath} % disable protrusion for tt fonts
}{}
\makeatletter
\@ifundefined{KOMAClassName}{% if non-KOMA class
  \IfFileExists{parskip.sty}{%
    \usepackage{parskip}
  }{% else
    \setlength{\parindent}{0pt}
    \setlength{\parskip}{6pt plus 2pt minus 1pt}}
}{% if KOMA class
  \KOMAoptions{parskip=half}}
\makeatother
\usepackage{xcolor}
\IfFileExists{xurl.sty}{\usepackage{xurl}}{} % add URL line breaks if available
\IfFileExists{bookmark.sty}{\usepackage{bookmark}}{\usepackage{hyperref}}
\hypersetup{
  pdftitle={ILO - Project Markdown},
  pdfauthor={Milo Skovfoged},
  pdfborder={0 0 0},
  breaklinks=true}
\urlstyle{same}  % don't use monospace font for urls
\usepackage[margin=1in]{geometry}
\usepackage{graphicx,grffile}
\makeatletter
\def\maxwidth{\ifdim\Gin@nat@width>\linewidth\linewidth\else\Gin@nat@width\fi}
\def\maxheight{\ifdim\Gin@nat@height>\textheight\textheight\else\Gin@nat@height\fi}
\makeatother
% Scale images if necessary, so that they will not overflow the page
% margins by default, and it is still possible to overwrite the defaults
% using explicit options in \includegraphics[width, height, ...]{}
\setkeys{Gin}{width=\maxwidth,height=\maxheight,keepaspectratio}
\setlength{\emergencystretch}{3em}  % prevent overfull lines
\providecommand{\tightlist}{%
  \setlength{\itemsep}{0pt}\setlength{\parskip}{0pt}}
\setcounter{secnumdepth}{-2}
% Redefines (sub)paragraphs to behave more like sections
\ifx\paragraph\undefined\else
  \let\oldparagraph\paragraph
  \renewcommand{\paragraph}[1]{\oldparagraph{#1}\mbox{}}
\fi
\ifx\subparagraph\undefined\else
  \let\oldsubparagraph\subparagraph
  \renewcommand{\subparagraph}[1]{\oldsubparagraph{#1}\mbox{}}
\fi

% set default figure placement to htbp
\makeatletter
\def\fps@figure{htbp}
\makeatother


\title{ILO - Project Markdown}
\author{Milo Skovfoged}
\date{27/4/2021}

\begin{document}
\maketitle

\hypertarget{analysis-of-fatalities}{%
\section{Analysis of Fatalities}\label{analysis-of-fatalities}}

\hypertarget{import-clean-and-merged-data}{%
\subsection{Import, clean, and merged
data}\label{import-clean-and-merged-data}}

First we download the Fatility data. We download the normalized data per
100.000 worker (INJ\_FATL\_ECO\_RT\_A)

Then we download the labor inspection data. Here we download the
different datasets:

\begin{enumerate}
\def\labelenumi{\arabic{enumi}.}
\tightlist
\item
  Number of labor inspectors per 10.000 worker (LAI\_INDE\_NOC\_RT\_A)
\item
  Number of inspections per inspector (LAI\_VDIN\_NOC\_RT\_A)
\item
  Total number of inspections (LAI\_VIST\_NOC\_NB\_A)
\end{enumerate}

\begin{verbatim}
##    ref_area         LaborInspectorsNormP10K      time      ref_area.label    
##  Length:454         Min.   : 0.0300         Min.   :2008   Length:454        
##  Class :character   1st Qu.: 0.3400         1st Qu.:2011   Class :character  
##  Mode  :character   Median : 0.6300         Median :2013   Mode  :character  
##                     Mean   : 0.9368         Mean   :2014                     
##                     3rd Qu.: 1.0425         3rd Qu.:2017                     
##                     Max.   :31.5600         Max.   :2019
\end{verbatim}

\includegraphics{ILO---Project-Markdown_files/figure-latex/DownloadLaborInspector-1.pdf}

\begin{verbatim}
##    ref_area         InspectionsPerInspector      time      ref_area.label    
##  Length:472         Min.   :  0.12          Min.   :2008   Length:472        
##  Class :character   1st Qu.: 37.40          1st Qu.:2011   Class :character  
##  Mode  :character   Median : 88.67          Median :2013   Mode  :character  
##                     Mean   :116.41          Mean   :2014                     
##                     3rd Qu.:161.70          3rd Qu.:2017                     
##                     Max.   :829.98          Max.   :2019
\end{verbatim}

\includegraphics{ILO---Project-Markdown_files/figure-latex/DownloadLaborInspector-2.pdf}

\begin{verbatim}
##    ref_area         TotalLaborInspections      time      ref_area.label    
##  Length:523         Min.   :     0        Min.   :1996   Length:523        
##  Class :character   1st Qu.:  2158        1st Qu.:2010   Class :character  
##  Mode  :character   Median : 17572        Median :2013   Mode  :character  
##                     Mean   : 51529        Mean   :2013                     
##                     3rd Qu.: 56350        3rd Qu.:2016                     
##                     Max.   :963443        Max.   :2019
\end{verbatim}

\includegraphics{ILO---Project-Markdown_files/figure-latex/DownloadLaborInspector-3.pdf}

Them we combine and clean the data.

\begin{verbatim}
## Joining, by = c("ref_area", "time", "ref_area.label")
## Joining, by = c("ref_area", "time", "ref_area.label")
## Joining, by = c("ref_area", "time", "ref_area.label")
\end{verbatim}

An initial linear regission analysis explains Fatilities very little
(R-squared: 0.0362)

It indicates a very small statical significants. The number of
inspections per inspector increases the number of Fatilities

However, the

\begin{verbatim}
## 
## Call:
## lm(formula = TotalFatilitiesNormP100K ~ LaborInspectorsNormP10K * 
##     InspectionsPerInspector, data = CombinedDataNoNA)
## 
## Residuals:
##    Min     1Q Median     3Q    Max 
## -5.288 -3.065 -0.615  1.532 33.000 
## 
## Coefficients:
##                                                  Estimate Std. Error t value
## (Intercept)                                      5.215961   0.564704   9.237
## LaborInspectorsNormP10K                          0.046517   0.118366   0.393
## InspectionsPerInspector                          0.009064   0.004556   1.990
## LaborInspectorsNormP10K:InspectionsPerInspector -0.014284   0.007563  -1.889
##                                                 Pr(>|t|)    
## (Intercept)                                     8.34e-16 ***
## LaborInspectorsNormP10K                           0.6950    
## InspectionsPerInspector                           0.0488 *  
## LaborInspectorsNormP10K:InspectionsPerInspector   0.0612 .  
## ---
## Signif. codes:  0 '***' 0.001 '**' 0.01 '*' 0.05 '.' 0.1 ' ' 1
## 
## Residual standard error: 4.929 on 125 degrees of freedom
## Multiple R-squared:  0.0362, Adjusted R-squared:  0.01307 
## F-statistic: 1.565 on 3 and 125 DF,  p-value: 0.2012
\end{verbatim}

\begin{verbatim}
## `geom_smooth()` using formula 'y ~ x'
\end{verbatim}

\includegraphics{ILO---Project-Markdown_files/figure-latex/PlotFatalies-1.pdf}

\begin{verbatim}
## `geom_smooth()` using formula 'y ~ x'
\end{verbatim}

\begin{verbatim}
## Warning in qt((1 - level)/2, df): NaNs produced

## Warning in qt((1 - level)/2, df): NaNs produced

## Warning in qt((1 - level)/2, df): NaNs produced

## Warning in qt((1 - level)/2, df): NaNs produced

## Warning in qt((1 - level)/2, df): NaNs produced
\end{verbatim}

\begin{verbatim}
## Warning in max(ids, na.rm = TRUE): no non-missing arguments to max; returning -
## Inf

## Warning in max(ids, na.rm = TRUE): no non-missing arguments to max; returning -
## Inf

## Warning in max(ids, na.rm = TRUE): no non-missing arguments to max; returning -
## Inf

## Warning in max(ids, na.rm = TRUE): no non-missing arguments to max; returning -
## Inf

## Warning in max(ids, na.rm = TRUE): no non-missing arguments to max; returning -
## Inf
\end{verbatim}

\includegraphics{ILO---Project-Markdown_files/figure-latex/PlotFatalies-2.pdf}

\begin{verbatim}
## `geom_smooth()` using formula 'y ~ x'
\end{verbatim}

\includegraphics{ILO---Project-Markdown_files/figure-latex/PlotFatalies-3.pdf}

\begin{verbatim}
## `geom_smooth()` using formula 'y ~ x'
\end{verbatim}

\begin{verbatim}
## Warning in qt((1 - level)/2, df): NaNs produced
\end{verbatim}

\begin{verbatim}
## Warning in qt((1 - level)/2, df): NaNs produced

## Warning in qt((1 - level)/2, df): NaNs produced

## Warning in qt((1 - level)/2, df): NaNs produced

## Warning in qt((1 - level)/2, df): NaNs produced
\end{verbatim}

\begin{verbatim}
## Warning in max(ids, na.rm = TRUE): no non-missing arguments to max; returning -
## Inf

## Warning in max(ids, na.rm = TRUE): no non-missing arguments to max; returning -
## Inf

## Warning in max(ids, na.rm = TRUE): no non-missing arguments to max; returning -
## Inf

## Warning in max(ids, na.rm = TRUE): no non-missing arguments to max; returning -
## Inf

## Warning in max(ids, na.rm = TRUE): no non-missing arguments to max; returning -
## Inf
\end{verbatim}

\includegraphics{ILO---Project-Markdown_files/figure-latex/PlotFatalies-4.pdf}

\hypertarget{including-plots}{%
\subsection{Including Plots}\label{including-plots}}

You can also embed plots, for example:

Note that the \texttt{echo\ =\ FALSE} parameter was added to the code
chunk to prevent printing of the R code that generated the plot.

\end{document}
